\section{Introduction}
The project is divided into 4 phases.

Phase 1 starts with cleaning N/A values and duplicates of the dataset, followed by feature importance analyzing and principle component analysis to reduce dimensionality. Normal transformation and standardization are applied to detect and remove outliers. Finally, correlation matrix suggest that collinearity does not exist in the dataset.

Phase 2 uses dataset generated by phase 1 to make predictions on a continuous numerical feature, number of installs, using a multiple linear regression model. As backward step-wise analysis failed to estimate meaningful regression model, random forest regressor proves to have a better fit based on MSE and $R^2$.

Phase 3 programmed a pipeline to analyze metrics and plot ROC curve of classifiers include decision tree, logistic regression, KNN, SVM, Na\"ive Bayes, Random Forest, Stacking, Boosting and Neural Network. The hyper parameter for each classifier is obtained by grid search with cross validation. Comparing the precision, recall, specificity, F score and ROC, the best model is Neural Network Classifier for performing better in AUC without compromising F score and precision.

Phase 4 uses K-means and DBSCAN to cluster the observations of the dataset and visualized the result using PCA. Optimal K is obtained by elbow method and silhouette analysis. Phase 4 also applies Apriori algorithm to find association rules among dataset.